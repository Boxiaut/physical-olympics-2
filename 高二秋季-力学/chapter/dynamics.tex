%!TEX root = ../xesphV.tex
\chapter{动力学}


\section{牛顿定律}

中世纪到近代以来物理学的发展,\,可谓沧海桑田,\,旧的观点被否定,\,几十年以后突然又以全新的面貌出现在物理学理论中.\,其中,\,实验中观察到各种令人惊诧的结果,\,无论物理学所依托的数学形式一变再变,\,要说亘古不变的主题,\,恐怕只有自然界的神奇与深邃,\,和我等凡人难以参透其中奥妙却仍为之孜孜不倦研究的决心.\,物理是一门研究自然界存在的运动并提出解释与理解方法的学科.\,运动形式多种多样,\,变化万千,\,但很多都被我们或多或少地解释,\,其实总结来看,\,从古至今我们也仅仅拥有过两类解释:

\begin{description}
	\item[牛顿开创的相互作用观:] \, 物质间存在相互作用,\,相互作用是改变物质运动状态的原因.\,这种观点植根于伽利略等先人对古希腊经验学说的批判.\,牛顿很成功地发展出系统的理论来,\,并很好地把天体的运动归结到互相之间产生的相互作用上.\,并表述为牛顿第二定律的形式.\,后人建立了场的理论以后,\,超距的粒子间相互作用就不再被理论学家们采纳了,\,而要理解为场与粒子的相互作用,\,再晚一点,\,场和粒子都被赋予了波粒二象性,\,粒子也要被理解为场,\,但所有场都要被被分解为平面波,\,平面波又要被量子化.\,如果没有相互作用,\,那么这些平面波的运动状态就不会发生改变,\,只有相互作用才会改变其运动,\,或传播方向发生偏折(发生散射),\,或强度发生改变(粒子的产生与湮灭).
	\item[爱因斯坦开创的几何观:] \, 
\end{description}

\section{动量定律}

\section{能量定律}

\section{角动量定律}

\section{位力定律*}

\section{动力学问题}

\section{碰撞问题}

