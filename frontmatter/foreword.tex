\chapter*{Foreword}
I am delighted to introduce the first book on Multimedia Data Mining.  When I came to know about this book project undertaken by two of the most active young researchers in the field, I was pleased that this book is coming in early stage of a field that will need it more than most fields do.  In most emerging research fields, a book can play a significant role in bringing some maturity to the field.  Research fields advance through research papers.  In research papers, however, only a limited perspective could be provided about the field, its application potential, and the techniques required and already developed in the field.  A book gives such a chance.  I liked the idea that there will be a book that will try to unify the field by bringing in disparate topics already available in several papers that are not easy to find and understand.  I was supportive of this book project even before I had seen any material on it.  The project was a brilliant and a bold idea by two active researchers.  Now that I have it on my screen, it appears to be even a better idea.  

Multimedia started gaining recognition in 1990s as a field.  Processing, storage, communication, and capture and display technologies had advanced enough that researchers and technologists started building approaches to combine information in multiple types of signals such as audio, images, video, and  text.  Multimedia computing and communication techniques recognize correlated information in multiple sources as well as insufficiency of information in any individual source.    By properly selecting sources to provide complementary information, such systems aspire, much like human perception system, to create a holistic picture of a situation using only partial information from separate sources.

Data mining is a direct outgrowth of progress in data storage and processing speeds.  When it became possible to store large volume of data and run different statistical computations to explore all possible and even unlikely correlations among data, the field of data mining was born.  Data mining allowed people to hypothesize relationships among data entities and explore support for those.  This field has been put to applications in many diverse domains and keeps getting more applications.  In fact many new fields are direct outgrowth of data mining and it is likely to become a powerful computational tool.\vadjust{\vfill\pagebreak}


