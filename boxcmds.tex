\renewcommand{\emph}[1]{{\color{red}\hei #1}}
\newcommand{\bs}[1]{\boldsymbol{#1}}
\newcommand{\pow}[1]{\times 10^{#1}}
\newcommand{\ap}{\textquotesingle}%文本内的prime
\newcommand{\ud}{\textrm{d}}
\newcommand{\uD}{\textrm{D}}
\newcommand{\ue}{{\rm e}}
\newcommand{\ui}{{\rm i}}
\newcommand{\uj}{{\rm j}}
\newcommand{\ke}{\frac{1}{4\pi\varepsilon_0}}
\newcommand{\kb}{\frac{\mu_0}{4\pi}}
\newcommand{\ca}{\raisebox{0.5mm}{------}\,}
\newcommand{\dbar}{\textrm{\dj}}

\DeclareMathOperator{\sh}{sh}
\DeclareMathOperator{\ch}{ch}
\let\th\relax
\DeclareMathOperator{\th}{th}
\DeclareMathOperator{\cth}{cth}
\DeclareMathOperator{\arccot}{arccot}
\DeclareMathOperator{\arsh}{arsh}
\DeclareMathOperator{\arch}{arch}
\DeclareMathOperator{\arth}{arth}
\DeclareMathOperator{\arcth}{arcth}
\DeclareMathOperator*{\res}{\mathfrak{Res}}
\newcommand{\pd}[3]{\left(\frac{\partial {#1}}{\partial {#2}}\right)_{#3}}
\DeclarePairedDelimiterX\norm[1]{\lVert}{\rVert}{#1}
\DeclarePairedDelimiterX\inp[2]{\lparen}{\rparen}{#1,#2}
\DeclarePairedDelimiter\bra{\langle}{\rvert}
\DeclarePairedDelimiter\ket{\lvert}{\rangle}
\DeclarePairedDelimiter\av{\langle}{\rangle}
\DeclarePairedDelimiterX\braket[2]{\langle}{\rangle}{#1 \,\delimsize\vert\, #2}
\DeclarePairedDelimiterX\opav[3]{\langle}{\rangle}{#1 \,\delimsize\vert\,\mathopen{}#2\,\delimsize\vert\,\mathopen{}#3}