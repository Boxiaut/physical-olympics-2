%!TEX root = ../../Physics_XES_II.tex
\chapterauthor{陈博}{学而思物理竞赛团队}
%\chapterauthor{Second Author}{Second Author Affiliation}
\chapter{Basic Concepts}

章节概述引入

\section{章节引入}\label{intro}
基本概念

\begin{enumerate}[1.]
\item 1

\item 2
\begin{enumerate}
\item 2.1

\item 2.2
\begin{enumerate}[iv.]
\item 2.2.1

\item 2.2.2

\end{enumerate}

\item 2.3

\item 2.4

\end{enumerate}

\item 3

\item 4


\end{enumerate}

\subsection{引入节}
faulty \cite{ilyas2004hsn}?


\begin{VF}
quote info

\VA{sometext}{anothertext}
\end{VF}


\begin{table}[H]
%\noautomaticrules
\tabletitle{engaged $(a_g^a)$  in a great civil war}%
\begin{tabular}{lccc}
\tch{Scene}    &\tch{Reg. fts.} &\tch{Hor. fts.} &\tch{Ver. fts.}\\
Ball &19, 221 &4, 598   &3, 200\\
Pepsi$^a$&46, 281 &6, 898 &5, 400\\
Keybrd$^b$   &27, 290 &2, 968 &3, 405\\
Pepsi    &14, 796 &9, 188 &3, 209\\
\end{tabular}
\end{table}

\textbf{MultiRelational $k$-Anonymity.} Most works called \emph{MultiR $k$-anonymity} to  $Y=\{Pid\}$.



\begin{shadebox} 
This doesn’t work with the pdftex engine -- see the manual?
\end{shadebox}


\begin{extract}
A component part for an electronic item is \cite{hyvarinen2001ica}
manufactured at one of three different factories.
\end{extract}

sometext.

\begin{equation}
\mbox{var}\widehat{\Delta} = \sum_{j = 1}^t \sum_{k = j+1}^t
\mbox{var}\,(\hat{\alpha}_j - \hat{\alpha}_k)  = \sum_{j = 1}^t
\sum_{k = j+1}^t \sigma^2(1/n_j + 1/n_k). \label{2delvart2}
\end{equation}


An obvious measure of imbalance is just the difference in the
number of times the two treatments are allocated
\begin{equation}
D_n = \mathcal{M}|n_A - n_B|. \label{2deffD}
\end{equation}
For rules such as deterministic allocation, for which the expected
value of this difference can be calculated, we obtain the population
value ${\cal D}_n$.

\begin{shortbox}
\Boxhead{Box Title Here}
Section~\ref{intro} studies this family of privacy models.

Let $m$ be a prime number. With the addition and multiplication as
defined above, $Z_m$ is a field.
\end{shortbox}

\begin{theorem}\label{1th:Z_m}
Let $m$ be a prime number. With the addition and multiplication as
defined above, $Z_m$ is a field.
\end{theorem}

\begin{proof}
Most of the proof of this theorem is routine.  It is clear that $0\in Z_m$
and $1\in Z_m$ are the
zero element and identity element. 
\end{proof}



\section{stuff}\label{sec:reclinkage}

In the privacy attack of \emph{record linkage}, some value $qid$ on $QID$ identifies a small number of records in the released table $T$, called a \emph{group}. 


\begin{table}
    \tabletitle{Examples for illustrating attacks}
    \begin{tabular}{|c|c|c|c|}
        \hline
        \textbf{Job} & \textbf{Sex} & \textbf{Age} & \textbf{Disease} \\
        \hline
        Engineer & Male & 35 & Hepatitis \\
        Engineer & Male & 38 & Hepatitis \\
        Lawyer & Male & 38 & HIV \\
        Writer & Female & 30 & Flu \\
        Writer & Female & 30 & HIV \\
        Dancer & Female & 30 & HIV \\
        Dancer & Female & 30 & HIV \\
        \hline
    \end{tabular}
    \label{table:rawpatient}
\end{table}



\subsection{stuff}
A component part for an electronic item is


\begin{figure}[htb]
\includegraphics[width=200pt]{chapters/chapter1/figures/cat.eps}
\caption[Short figure caption]{very long captions}
\end{figure}





\begin{figure}
\centering
\subfigure[\label{f8a}]{\includegraphics[width=7cm]{chapters/chapter1/figures/Histogram.eps}}
\subfigure[\label{f8b}]{\includegraphics[width=7cm]{chapters/chapter1/figures/Histogram.eps}}
\caption[titleinmenu]{VERYLONG}
\end{figure}

\subsubsection{subsub}
A component part for an electronic item is
\begin{enumerate}
\item[\rm (a)] For all $u$ and $v$ in $V$, $u+v$ is
also in $V$.
\item[\rm (b)] For all $u$ in $V$ and $c$ in $F$, $cu$ is
in $V$.
\end{enumerate}


The zero vector itself is a subspace.

\begin{definition}\label{1def:linearcomb}{\rm
d a {\it linear combination} \index{linear combination}

The vectors $u^{(1)},u^{(2)},\ldots,u^{(m)}$ {\it span} $V$ \index{spanning set}
(equivalently, form a {\it spanning set} of $V$) 

\hfill{$\Box$}
}\end{definition}

over

\begin{VT1}

\VH{Think About It...}

Commonly  dime.

\VT
With the continual expansion 

\VTA{same}{same before}
\end{VT1}


\section{Glossary}
\begin{Glossary}
\item[glossary] entry1
\item[entry2] exp
\item[...] ...
\end{Glossary}



