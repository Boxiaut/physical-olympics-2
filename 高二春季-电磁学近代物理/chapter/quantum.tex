%!TEX root = ../xesphVII.tex
\chapter{量子论}

我们用选自\emph{费曼}(R. Feynman)先生的物理学讲义的开篇金句来作为本章与下一章近代物理内容的开头:
\begin{quote}
Each piece, or part, of the whole of nature is always merely an approximation to the complete truth. Therefore, things must be learned only to be unlearned again or, more likely, to be corrected. The test of all knowledge is \emph{experiment}. Experiment is the sole judge of scientific ``truth''.
\end{quote}

的确,\,量子理论以其不直观而被近代早期物理学家们所疑惑,\,这其中不乏一些赫赫有名的大师.\,直到现在也有很多基础的问题是没有被深刻地理解的:\,电子的本性与内部结构,\,基本粒子的类别与参数,\,对量子非定域性与测量的理解...\,所以在关于可能会造成问题的领域的学习时,\,采取先明白实验上的事实,\,从中理解理论建立的必要性.


\section{黑体辐射}

对于\emph{热辐射}(thermal radiation)的讨论是何时进入物理研究的视野的呢?\,可以肯定的是人类认识到热辐射现象非常的早:\,光芒万丈的太阳,\,烧红的木炭与金属都是典型的热辐射的情形.\,但人们掌握足够的方法去测量它则也是要到19世纪后半期了.\,热辐射势必涉及到电磁场与电荷的相互作用.\,而且深入到原子尺度,\,实际上就是电磁波的发射与吸收.\,对于电磁波的发射,\,我们在电磁学中粗略讲过,\,只要有加速运动的电荷就会导致电磁辐射.\,之后小节我们将认识到微观电荷不能用``加速''来描述,\,其状态其实是量子态,\,处于激发态才会自发向基态去跃迁放出电磁辐射.\,而对于电磁波的吸收,\,则在光学中我们简要介绍过洛伦兹电子论中的处理方法.\,

一个物体如果能够在任何温度下把照射到它上面的任何频率的光都全部吸收掉,\,那么这个物体就叫做\emph{黑体}(black body).\,虽然在现实生活中这样的物体并不存在,\,但石墨和碳黑往往被视作比较理想的黑体,\,尽管这样,\,黑体``看上去''也不总是``黑的''.\,这里有两个主要的原因:




\section{光粒子性}

\section{玻尔原子}

\section{电子波动性}

\section{物质波与波函数}

