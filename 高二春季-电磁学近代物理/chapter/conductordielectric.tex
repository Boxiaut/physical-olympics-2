%!TEX root = ../xesphVII.tex
\chapter{导体与介质}




\section{导体与静电平衡}

\subsection{导体特点}

微观地看,\,物质由原子或分子等组成,\,其中导电现象通常发生在不同情况下:
\begin{enumerate}
	\item 真空导电:\,一般不会说真空具有\emph{导电性}(conductive),\,因为真空中是没有\emph{载流子}(charge carrier)的.\,的确,\,在\emph{阴极射线管}(cathode ray tube)中,\,加热一个阴极,\,并辅以合适的偏置电压可以造成真空中的电流.\,或是用一束能量足够的光子去轰击金属表面造成电子逸出,\,甚至纯粹由于阴极表面尖处十分强的电场导致电子直接克服逸出功发射出来.\,三种现象分别称为\emph{热发射}(thermionic emission),\,\emph{光电效应}(photoelectric effect)与\emph{场发射}(field emission).这种电荷的定向移动现象被统一地称为\emph{输运现象}(transport phenomenon).\,由于真空输运的独特性质,\,比如电子不会受到散射,\,平均自由程远大于仪器尺寸,\,与凝聚态物理中的一些概念对应,\,这被称为\emph{弹道输运}(ballistic transport).
	\item 绝缘体漏电:\,大多数非金属晶体,\,或是不含离子的液体与气体,\,原子核在晶体中都限制在点阵格子的特定位置,\,在气体,\,液体中则没有特定位置.\,电子分为两类,\,,一类是原子的内层电子,\,它极为稳定地存在于原子周围,\,离子晶体中的几乎完全被阴离子夺取的电子也属于这种情况.\,它们与原子核一起构成\emph{原子实}(atomic core).\,而\emph{价层电子}(valence shell electron),\,它们用来成键,\,将原子连接形成分子或者晶体,\,一般被定域在原子间的特定区域.\,所有这些电子的特点都是\emph{束缚态}(bond state).\,它们不能在介质中自由传导,\,一个电子的区域到另一个电子的区域间存在\emph{势垒}(potential barrirer),\,经典物理认为电子的动能不足以穿过这些势垒.\,但是,\,量子理论则认为电子的波函数可以通过\emph{隧道效应}(tunnelling)以小概率在不同区域间转移.\,这就为有电场的情况下电荷的平均定向移动提供了可能.\,这种现象称为\emph{漏电}(leakage).
	\item 绝缘体击穿:\,在十分高的电场下,\,原子与原子间的典型电压将大于阻碍电子转移的势垒,\,或是分子中显不同电性的部分之间.\,此时电子在本质上可以视为在
\end{enumerate}

\section{电像法}

\section{电介质}

\section{再议静电能}
