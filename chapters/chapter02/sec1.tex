%!TEX root = ../../Physics_XES_II.tex
%\chapterauthor{\sf 认识与描述物质的世界...}{}
%\chapterauthor{Second Author}{Second Author Affiliation}
\chapter{运动学}\label{2}

\emph{插图}



\section{时空与物质}\label{2-1}

物理学, 从刚开始成为实验性的科学的伽利略时期开始, 到近半个世纪年来理论物理学家对额外维度的探讨, 都给予了\emph{时空}(spacetime)最核心的地位. 牛顿的理论, 分析力学, 经典场论, 相对论这些理论最基本的图像都是时空与\emph{物质}(matter)的分立性. 时空是装备了一个能体现出物理物理意义的\emph{度量}(metric)的3+1维对象. 数学上有一套严格的说法, 把这种连续, 光滑的四维对象称为\emph{伪黎曼流形}(pseudo-Riemannian manifold). 而物质则是在每个时空点处的某种结构. 经典理论下, 这种结构不外乎用\emph{标量}(scalar), \emph{矢量}(vector)或是更高级的\emph{张量}(tensor)来描述. 最后, 这些量的变化就体现出了形形色色的\emph{运动}(motion). \emph{运动学}(kinematics)的主要任务就是描写物质的运动, 而对于物质运动背后体现出来的\emph{定律}(law), 我们仅做笼统的较浅的讨论, 更加深入地看, 不同物质运动符合的不同定律往往又需要因为时空的结构或相互作用的内禀属性而具有普遍的\emph{对称性}(symmetry), 对称性是``规律之规律'', 它对物理理论在何种程度上起到决定作用将是超出本书范围的更高层次的课题, 将伴随读者对物理学学习的生涯.

\subsection{时空观}\label{2-1-1}

牛顿力学理论体系基于\emph{伽利略时空观}(Galilean spacetime picture), 也就是\emph{绝对时空观}(absolute spacetime picture). 在这里空间是三维的平直空间. 设想一个人站在该空间的某个空间点, 他的胸前, 头顶和右手平举的三个方向就是相互垂直的方向. 如果以自己的臂长为标准长度, 他能够定义空间中每两个点之间的空间间隔. 事实上, 这个人能够以一种正确的方式为每个空间点定义一个坐标, 那么所有三维空间点的集合与任意两个点之间的距离为:

\[A(x,\,y,\,z)\in\mathbb{R}^3 \quad;\quad x,\,y,\,z\in\mathbb{R}\]
\[A(x_1,\,y_1,\,z_1)\;,\;B(x_2,\,y_2,\,z_2)\;:\;l^2=\overline{AB}^2=(x_1-x_2)^2+(y_1-y_2)^2+(z_1-z_2)^2\]

因为空间不同于时间, 我们为以上数字赋予特殊的物理含义, 也就是加上\emph{量纲}(dimension). 如果两点坐标为$(0,\,0,\,0)$和$(1,\,1,\,1)$, 那么$l= \si{\sqrt{3}m}$, 而不是$l=\sqrt{3}$. 其单位$\si{m}$一方面表示了这个物理量的属性, 另一方面指定了某个实际物理体系确定下来的固有长度大小. 现行的(2018年1月1日, 下文同)国际单位制对$\si{1m}$的定义如下\footnote{一方面, 它依赖于狭义相对论的正确性, 目前极少理论物理工作者会质疑它. 另一方面, 应该要先定义下文的$\si{1s}$, 再来定义$\si{1m}$.}:
\begin{verse}\sf\large
$\si{1m}$是光在$\si{\frac{1}{299792458}s}$内在真空中行进的距离.
\end{verse}


这就是我们的\emph{三维平直空间}(3-dimensional flat space). 注意空间点具有物理实际意义, 它可以脱离坐标系而单独存在. 事实上坐标系的原点可以建立在空间中的任意点处, 朝向也可以是任意方向, 两个空间点之间的距离$l$不会依赖于坐标系的选取, 但两个点的坐标会因坐标系不同而改变. 如果我们选取的坐标系总是下述使得两个相距很近的点之间的微元距离公式成立:
\[\ud l^2=\ud x^2+\ud y^2+\ud z^2\]


那么建立的坐标系就是一个\emph{笛卡尔坐标系}(Cartesian coordinate system), 即\emph{空间直角坐标系}(3D orthogonal coordinate system). 但是以空间直角坐标系为基础, 我们又经常建立\emph{球坐标}(spherical coordinate)和\emph{柱坐标}(cylindrical coordinate)系统. 通常取$x$轴为\emph{幅轴}(azimuth axis), 点在$x-y$平面上的投影与原点的连线相对$x$轴转过的角度为$\varphi$, 即\emph{幅角}(azimuth angle). 而$z$轴为\emph{极轴}(polar axis), 而点与原点连线与极轴的夹角$\theta$为\emph{极角}(polar angle). 天文观测用球坐标就很方便, 它是以描述的空间点到原点之间的距离$r$, 也称\emph{矢径}(radius), 和两个描述角位置的极角幅角来构成三个坐标$(r,\,\theta,\,\varphi)$的. 而理论物理里也常用到的柱坐标是以$(\rho,\,\varphi,\,z)$为描述空间点的坐标, $\rho$是空间点到$z$轴的距离.

\begin{wrapfigure}[16]{o}[-10pt]{7cm}
\vspace{-0.4cm}
\centering
\includegraphics[width=7cm]{2.1.png}
\caption{三种坐标}
\end{wrapfigure}
之后经常会说到各种对称性, 在本系列教材中我们采取如下说法: \emph{球对称}(spherical symmetric)仅仅代表某个函数$f(r,\theta,\varphi)$与$\varphi$无关. 而\emph{柱对称}(cylindrical symmetric)代表的是$f(\rho,\,\varphi,\,z)$与$z$无关. 与$\varphi$和$\theta$都无关的$f(r,\,\theta,\,\varphi)=f(r,\,\forall\theta,\,\forall\varphi)$被称为\emph{各向同性}(isotropic). 最后\emph{中心对称}(centrosymmetric)是一个很弱的对称性, 它仅仅代表函数在\emph{中心反演}(space inversion)下的对称性:
\[f(x,\,y,\,z)=f(-x,\,-y,\,-z)\]

绝对时空观中的时间则是一种完全与空间独立的属性. 任意一个坐标点处都有时间轴, 而任意一个时刻都有一个三维空间切片. 事实上, 我们的时空是一个3+1维的结构, 合理地选取坐标后, 实际上可以把时空结构写成四维坐标:
\[A(x,\,y,\,z,\,t)\in\mathbb{R}^4 \quad;\quad x,\,y,\,z,\,t\in\mathbb{R}\]

时间是一个新的量纲, 其国际单位制对$1{\rm s}$的定义为:
\begin{verse}\sf\large
\si{1s}是$^{133}{\rm Cs}$原子基态的两个超精细能级之间跃迁所对应的辐射周期时长的$9192631770$倍.
\end{verse}

而对于任意两个时空点, 存在绝对的时间间隔, 也就是可以找到两个事件的绝对时间差:
\[\tau=|t_2-t_1|\]

但空间距离却具有相对性. 故我们要求在同一时刻的空间切片上定义空间的度量:
\[t_1=t_2:\quad l=\sqrt{(x_1-x_2)^2+(y_1-y_2)^2+(z_1-z_2)^2}\]

这种时空结构被叫做\emph{牛顿-嘉当几何}(Newton-Cartan geometry). 在牛顿-嘉当几何中书写的物理规律应当具有时空平移对称性, 空间旋转对称性和\emph{伽利略变换}(Galilean transformation)的对称性, 此后我们将进一步阐述. 时空平移是指将原来发生的物理过程随时空坐标进行如下改变:
\[\begin{cases}t \quad &\longrightarrow \quad t+\Delta t\\x \quad &\longrightarrow \quad x+\Delta x\\y \quad &\longrightarrow \quad y+\Delta y\\z \quad &\longrightarrow \quad z+\Delta z \end{cases} \]

空间旋转对称性的一种简单情况是绕$z$轴旋转$\theta$角:
\[\begin{cases}t \quad &\longrightarrow \quad t\\x \quad &\longrightarrow \quad x\cos\theta-y\sin\theta\\y \quad &\longrightarrow \quad y\cos\theta+x\sin\theta\\z \quad &\longrightarrow \quad z\end{cases} \]

伽利略变换是指不改变过程发生的时间, 但是对于不同时刻发生的事件的坐标进行重新标定, 从而使得两个坐标系之间恰好只差一个速度为$\bs{v}=(v_x,\,v_y,\,v_z)$的匀速直线运动:
\[\begin{cases}t \quad &\longrightarrow \quad t\\x \quad &\longrightarrow \quad x-v_x t\\y \quad &\longrightarrow \quad y-v_y t\\z \quad &\longrightarrow \quad z-v_z t \end{cases} \]




不同于绝对时空观中时间与空间成为相互独立的量纲的特点, 狭义相对论改变了对基本物理量的看法. 狭义相对论把时空看成为可以相互转化的不可分割的新的3+1维整体. 两个时空点之间无法定义绝对的时间间隔. 在\emph{相对论时空观}(relativistic spacetime picture)下, 时间和空间可以用同一把尺子去丈量: 这是由于光速的不变性:
\[c=299792458{\rm m/s}\]

而量出来的长度叫做\emph{时空间隔}(spacetime interval):
\[\ud s^2=c^2\ud t^2-\ud x^2-\ud y^2-\ud z^2\]

这赋予时空以截然不同的结构, 最关键的一点是, 时间顺序的绝对性被取消了, 相对论的有限速度因果律在这里取代了经典观点的时序因果律. 相对论理论我们将在此后章节展开介绍.

\subsection{物质观}\label{2-1-2}

时空是物理过程发生的舞台. 舞台上排演的剧目则是各式各样的\emph{物质}(matter)作为演员, 而根据其间的\emph{相互作用}(interaction)作为剧本而造就的\emph{可观测量}(observable)的变化. 任何物理理论, 描述物质的存在形式, 描述相互作用的形式当然以某些含\emph{物理量}(physical quantity)的公式作为数学载体. 但是, 不是所有物理量都一定是可观测量. 可观测量, 即可以测量的物理量, 读者也许会对这个概念感到陌生, 难道还有不可以测量的物理量吗? 当然有, 以下是两个例子:
\begin{itemize}
	\item 研究交流电问题中, 将交流电流升格为相量(复数):
	\[i=I_0\cos(\omega t+\varphi) \quad \longrightarrow \quad \widetilde{I}=\frac{I_0}{\sqrt{2}}\ue^{\uj \left(\omega t+\varphi\right)}\]

	则虚部$\mathfrak{Im}\widetilde{I}$由于是人为引入的辅助性量而不是可观测量.

	\item 研究静电学问题中, 由于不一定选取无穷远为电势的零点, 故实际上一个静电学体系在某点产生的电势为:
	\[\varphi=\int\frac{\ud Q}{4\pi\varepsilon_0 r}+C\]

	所以同一个体系可能具有不同的$\varphi$, 即用$\varphi$去描述物理体系时存在冗余. 从而电势的绝对大小不是可观测量, 两个点电势的差值才是可观测量.

	\item ...

\end{itemize}

对于一个物理理论来说, 基本定律中的物理量不一定是可观测量, 此时应当给出利用物理量计算可观测量的额外公式. 所幸在本书的大部分篇幅中, 各类物理量都是可以进行测量的. 故仅在遇到不可观测量时例外强调.

物理量可以分为两类: 第一类是时空坐标$\bs{r},\,t$, 它是物质运动在时空上的外化. 第二个类描述物质内禀的属性. 它取决于所引入的物质种类, 还取决于我们所关心问题的层次. 一般用\emph{标量}(scalar), \emph{矢量}(vector), 乃至\emph{张量}(tensor), \emph{旋量}(spinor)这样的数学工具来描述它. 举例, 电磁场物质在经典物理中用电场磁场来描述, 但在\emph{量子力学}(quantum mechanics,  QM)中这不够了, 需要用矢势和标势来描述才是完整的. 在更深的\emph{量子场论}(quantum field theory,  QFT)中甚至这也是不够的. 还需要量子化为光子才合适. 又比如, 电子参与的物理现象, 最简单的电子模型是质点模型, 其内禀的属性是质量, 动量与能量. 然而与电磁场的经典相互作用强度告诉我们还有一项内禀属性叫做电荷量. 近代人们惊奇地发现原来电子还固有磁矩, 与之相应的电子具有内禀的自旋. 最后狄拉克等人发展出\emph{量子电动力学}(quantum electrodynamics, QED), 统一地用一个四分量的旋量波函数就能完整地描述所有发现的电子内禀属性.


我们来看一些典型的关于物质的理论模型:

\subsubsection{质点模型}

不得不承认质点是牛顿力学的根基, 一切可观的结论的出发点. 质点所对应的事件集合为时空中的一条\emph{世界线}(world line). 每一个时间仅仅有可能只有一个事件发生. 实际上质点的运动用\emph{运动学方程}(kinematic equation)来描述:
\[\bs{r}=\bs{r}(t)=\left(x(t),\,y(t),\,z(t)\right)\]

而\emph{质量}(mass)是质点必要的内禀属性. 它将作为参数出现在下一节介绍的动力学方程中. 它反应物质受到同样大小的相互作用运动状态改变的难易程度. 国际单位制从1889年至2018年末对$1\rm kg$的定义如下:

\begin{verse}
1kg是保存在法国巴黎布勒特伊宫的国际计量局实验室的约47立方厘米立式铂铱合金小圆柱的质量. 当然, 出于实用考虑, 也是很多它的官方复制体的质量.
\end{verse}

这个定义今已经有一百多年了, 历史远长于其他六个国际基本单位. 但目前这一古老的定义方式已经被废除, 新的定义\footnote{一方面, 它依赖于狭义相对论和量子力学的正确性, 目前极少理论物理工作者会质疑它. 另一方面, 应该要先定义上文的1s和1m, 再来定义1kg.}为:

\begin{verse}
1kg被这样定义: 取普朗克常数的固定数值在单位$\rm kg\cdot m^2\cdot s^{-1}$下为$\rm 6.62607015\times 10^{-34}$.
\end{verse}

长度, 时间和质量为三大力学量纲, 量纲是物理量的属性, 物理量的表示方法为数值加单位, 每个量纲有自己独特的一套单位, 不同单位间差一个纯数的倍率. 在物理量的加减时量纲必须相同而且结果保持量纲不变. 但不同量纲物理量可以进行乘除而生成新的量纲. 除了简单的加减乘除的以上规则以外, 其他特殊函数必须只能作用在无量纲的纯数上. 这叫做\emph{量纲法则}(dimensional rules).




\subsection{世界观}\label{2-1-3}