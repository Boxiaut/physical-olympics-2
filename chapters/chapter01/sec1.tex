%!TEX root = ../../Physics_XES_II.tex
\chapterauthor{阅读本书前需要明白的资料...}{}
%\chapterauthor{Second Author}{Second Author Affiliation}
\chapter{普通物理学概论}\label{1}

章节概述引入

\section{范畴与方法论}\label{1-1}

self-contained: 数学知识有一定基础后不需要更多的补充

dependency-requalified: 
\[\text{力学}>\text{热学}>\text{电磁学}>\text{光学}>\text{近代物理}\]

picture-oriented: 由于数理基础不够而导致的有关物理原理背后的理论基础造成困难的现象, 我们企图用``物理图像''帮助读者理解其结果的自然性, 这样的章节用星号来标注, 读者不应当忽略其重要性, 应当在基础足够以后重新阅读相关章节.

读者也许知道, 矢量形式的\emph{牛顿力学}(Newtonian mechanics)理论, 和在其后发展出来的\emph{分析力学}(analytic mechanics)理论, 可以统称为\emph{经典力学}(classical mechanics)理论. 它们将是本书第I部分---力学的研究范围\footnote{分析力学仅做引入与概述.}. 而\emph{经典物理学}(classical physics)本书专指在二十世纪之前成熟的物理理论, 它还需要包括\emph{经典热力学}(classical thermodynamics)理论, 我们将在本书的第II部分---热学介绍; \emph{经典电磁学}(classical electrodynamics)理论, 我们将在本书的第III部分---电磁学介绍; 和\emph{几何光学}(geometric optics)与\emph{波动光学}(wave optics)理论, 我们将在本书的第IV部分---光学介绍.