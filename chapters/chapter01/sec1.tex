%!TEX root = ../../Physics_XES_II.tex
\chapterauthor{阅读本书前需要明白的资料...}{}
%\chapterauthor{Second Author}{Second Author Affiliation}
\chapter{普通物理学概论}\label{1}

章节概述引入

\section{范畴与方法论}\label{1-1}

self-contained: 数学知识有一定基础后不需要更多的补充

picture-oriented: 读者应该注意到了, 本套普通物理学教材的五个部分: 力学, 电磁学, 近代物理, 热学, 光学的排列顺序与市面上的其他常见教材的顺序是不同的. 对于一套普通物理学教材来说, 先后顺序是个大问题: 章节之间依赖关系, 举例: 不将电磁学安排至热学前, 则讲热学时热电耦合现象讲解就会缺乏基础, 不将热学安排到电磁学前面, 则讲电磁学时德鲁特模型中电子运动的统计本性就难以充分展开说明. 尽可能多地排序使得能覆盖更多的知识模块, 对子学科的基础性进行排序, 更加基础的子学科先讲, 这样我们得到以下顺序:
\[\text{力学}<\text{电磁学}<\text{近代物理}<\text{热学}<\text{光学}\]

读者若进行更深层次的理论物理学研究, 就会发现这个顺序与后续理论物理学的学习顺序才是相似的. 这么安排也能帮助读者预知理论物理学的理论框架. 而对于我们选择的这种排序注定会产生的(尽管我们尽可能地去规避它, 带星号表示)若干重要物理学现象解释缺少基础的问题, 我们将通过...的方式充分调动读者对于形成正确物理学图像的直觉能力.