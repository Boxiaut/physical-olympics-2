%!TEX root = ../physical-olympics-2.tex
\chapter{振动与波}


\section{方程与谐振}

\subsection{简谐振动的定义}
振动是最常见的物理现象.\,而振动中的最简单(simple)而和谐(harmonic)者谓之\emph{简谐振动}(simple harmonic oscillation).\,对\emph{谐振子}(harmonic oscillator)的学习与研究是会贯彻整个物理理论不同层次内容的始终的.\,现在是经典力学,\,以后会上升到场论,\, 量子力学与量子场论的高度.

简谐振动是指一个物理量$Q$随时间围绕其平衡位置做上下的波动.\,其形式符合:
\[Q=Q_0+\Delta Q\cos(\omega t+\varphi)\]

我们经常会有用复数表示振动的习惯,\,其做法是在三角函数$\cos$与其\emph{宗量}(argument)\,$\phi=\omega t+\varphi$构成的项后添加一个虚的$\ui \sin\phi$项,\,于是新的写法变成:
\[\tilde{Q}=Q_0+\Delta Q\ue^{\ui(\omega t+\varphi)}\; ; \; Q=\mathfrak{Re}(\tilde{Q})\]

又或者:
\[\tilde{Q}=Q_0+\Delta \tilde{Q}\ue^{\ui\omega t}\; ; \; \Delta \tilde{Q}=\Delta Q\ue^{\ui\varphi}\]

以上各个常量中,\,$Q_0$是平衡位置,\,$\Delta Q$叫\emph{振幅}(amplitude),\,宗量$\omega t+\varphi$叫做\emph{相位}(phase),\,$\omega$叫\emph{角频率}(angular frequency),\,$\varphi$叫初相位.\,$\tilde{Q}$为复化的复物理量,\,而$\Delta \tilde{Q}$叫做\emph{复振幅}(complex amplitude).

复数表示最大的一个好处就在于很方便计算物理量的线性组合与导数.\,事实上:
\[\dot{Q}=-\omega\Delta Q\sin\phi\; ; \; \dot{\tilde{Q}}=\ui\omega\tilde{Q}\quad \Rightarrow\quad \dot{Q}=\mathfrak{Re}(\dot{\tilde{Q}})\]
\[\ddot{Q}=-\omega^2\Delta Q\cos\phi\; ; \; \ddot{\tilde{Q}}=-\omega^2\tilde{Q}\quad \Rightarrow\quad \ddot{Q}=\mathfrak{Re}(\ddot{\tilde{Q}})\]

\subsection{简谐振动的性质}
以一个质点水平坐标$x$围绕$x=0$左右做谐振为例.\,其运动方程写作\footnote{以后我们对物理量和物理量的复化只在必要的时候加以区分,\,看到复数只需要认为省写了取实部这一例常操作罢了.}:
\[x=A\ue^{\ui(\omega t+\varphi)}\]

那么其速度与加速度为:
\[v=\ui\omega A\ue^{\ui(\omega t+\varphi)},\,a=-\omega^2A\ue^{\ui(\omega t+\varphi)}\]

我们发现在运动过程中的任意一个时刻,\,加速度都是指向平衡位置的,\,与偏离平衡位置的位移是成正比的:
\[a=-\omega^2 x\]

而任意一个时刻既然$x$是宗量的余弦函数,\,$v$是正弦函数,\,它们就满足其绝对值大小的``此消彼长''关系\footnote{注意到这里要避免使用复数,\,因为一个复数的实部平方不会等于其平方的实部$v^2\neq\mathfrak{Re}(\tilde{v}^2)$}:
\[v^2+\omega^2 x^2=\omega^2 A^2\]

相位是很重要的物理概念,\,\emph{相}(phase),\,状态也,\,相位则是一个可以用来表示状态的数.\,反过来,\,我们可以根据物体在一个时刻的状态反过来确定这个时刻的相位.\,如果位移是$x$而振幅为$A$,\,则:
\[\phi=\mathrm{Arcsin}\frac{x}{A}\, ,\,\mathrm{Arcsin}\frac{x}{A}\in \left\{\arcsin\frac{x}{A}+2n\pi|n\in\mathbb{Z}\right\}\cup\left\{\pi-\arcsin\frac{x}{A}+2n\pi|n\in\mathbb{Z}\right\}\]

同理如果已知速度$v$和\emph{速度振幅}(velocity amplitude)\,$\omega A$,\,那么相位为:
\[\phi=\mathrm{Arccos}\frac{v}{\omega A}\, ,\,\mathrm{Arccos}\frac{v}{\omega A}\in \left\{\arccos\frac{v}{\omega A}+2n\pi|n\in\mathbb{Z}\right\}\cup\left\{-\arccos\frac{v}{\omega A}+2n\pi|n\in\mathbb{Z}\right\}\]

相位的取法上具有多值性.\,但是相位随时间的变化我们约定必须是连续的,\,事实上它随时间线性增加.\,也就是说如果前一个状态下相位为$\phi_1$,\,后一个状态下相位为$\phi_2$,\,那么这两个状态间历时:
\[\Delta t=\frac{\phi_2-\phi_1}{\omega}\]



\section{阻尼振动与受迫振动}

\section{多自由度小振动*}

\section{非线性摄动}

\section{格波}

\section{波动方程}

\section{波的色散}

