%!TEX root = ../physical-olympics-2.tex
\chapter{万有引力}


\section{有心力下运动}

天体运动中起到核心作用的相互作用力都是\emph{有心力}(central force),\,事实上不光是天体运动,\,任意两个可以近似为质点的物体之间的相互作用力,\,根据牛顿第三定律的要求,\,其受力方向都必须沿着两个物体的连线方向.\,那么我们只要做两个要求,\,这就构成了一个有心力问题:

一是,\,这个力必须是保守力,\,即,\,它必须由势能生成:
\[V(\bs{r}_1,\,\bs{r}_2):\quad\,\bs{F}_1=-\nabla_1V\;,\,; \bs{F}_2=-\nabla_2V\]

根据我们之前的说法,\,根据对称性或牛顿第三定律,\,这个势能其实就是两个质点连线距离$R$的函数$V(R)$.\,而以$1$为中心向$2$引$\bs{R}$矢量,\,则:
\[\bs{F}_2=-V'(R)\bs{e}_{\bs{R}}=F(R)\bs{e}_{\bs{R}}\]

二是,\,中心物体$1$必须不动或者是近似不动.\, $1$不动是指有外力作用在$1$上以维持其静止.\,但$2$上不应该有这样的外力,\,而仅仅是在$1$对$2$产生的$\bs{F}_2$作用下做运动.\,$1$近似不动是比如考虑太阳系这种典型情况,\,太阳虽然受到多个行星对它的万有引力,\,但是由于自己质量过重从而近似是不动的.\,即使是地球月亮构成的二体问题,\,地球和月亮并不一定能认为都不动,\,我们也有相应的转化为有心力问题的方法,\,见后.\,最后当然,\,也有一些更简单的情况,\,$2$受到一些更复杂的体系对它的力构成了有心力$F(R)$,\,比如弹性绳对绳端质点的拉力.



\section{万有引力下运动}

\section{二体与潮汐}

