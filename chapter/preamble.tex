%!TEX root = ../physical-olympics-2.tex
\chapter{序}

\section{说在一切前面}

\begin{altverse}
\centering \large

There are clues everywhere,\\
 all around us.\\
The wrong interpretatation of the clues,\\
 we call our world.\\
---  David Lynch\\
\end{altverse}

现代人的生活,\,变换莫测,\,日新月异.\,不同人用着不同的方式去理解变化背后体现出来的规律.\,优秀的理解方式互相交流与学习,\,学科就从中诞生.\,\emph{物理学}(physics)就是其中的一门学科,\,它是\emph{自然科学}(nature science)的主要构成成分之一.\,今天也许很难找到依据为物理学和自然科学的范围明确划分界限,\,但是物理学秉承了自然科学的一些重要方法论特征,\,是可以帮助我们明显的与其他学科做区分的,\,它们包括:
\begin{itemize}
\item 经验主义.\,把物理学和文学,\,艺术,\,乃至亲缘关系近的数学比对,\,就容易发现后者们的核心在于创新而非做回顾.\,而物理很多时候,\,etc
\item 实验物理.\,
\item 可证伪性.\,
\item 与数学水乳交融:\,描述与诠释.\,
\item 有效预言.\,
\end{itemize}


\section{本书编排}

\section{预备知识}

\subsection{力学}

\subsection{电磁学}

\subsection{热学}

\subsection{光学}

\subsection{近代物理}

\subsection{数学}
