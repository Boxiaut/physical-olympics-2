%!TEX root = ../physical-olympics-2.tex
\chapter{动力学}


\section{牛顿定律}

\subsection{概述}
中世纪到近代以来物理学的发展,\,可谓沧海桑田,\,旧的观点被否定,\,几十年以后突然又以全新的面貌出现在物理学理论中.\,其中,\,实验中观察到各种令人惊诧的结果,\,无论物理学所依托的数学形式一变再变,\,要说亘古不变的主题,\,恐怕只有自然界的神奇与深邃,\,和我等凡人难以参透其中奥妙却仍为之孜孜不倦研究的决心.\,物理是一门研究自然界存在的运动并提出解释与理解方法的学科.\,运动形式多种多样,\,变化万千,\,但很多都被我们或多或少地解释,\,其实总结来看,\,从古至今我们也仅仅拥有过两类解释:\,一类由牛顿提出,\,一类则来自爱因斯坦.\,前前后后经历了大概以下发展:

\begin{description}
	\item[亚里士多德:]\, 天体运行,\,自由落体是自然(nature)的运动;\, ``树欲静而风不止''是强迫(violent)的运动.\,但错误地认为``物体越重,\,下落越快''
	\item[伽利略:]\, 辩证地看待亚里士多德的观点,\,开创性地提出相对性原理,\,成功论证与形成惯性与常见运动的正确认识.
	\item[笛卡尔,\,惠更斯等:]\, 成功得到大量动力学公式,\,包括后来的牛顿第二定律,\,动能的表达式与能量守恒定律,
	\item[牛顿开创的相互作用观:] \, 开创性的提出物质间存在相互作用,\,相互作用是改变物质运动状态的原因.\,并很好地把天体的运动归结到互相之间产生的相互作用上.\,后人建立了场的理论以后,\,超距的粒子间相互作用就不再被理论学家们采纳了,\,而要理解为场与粒子的相互作用,\,再晚一点,\,场和粒子都被赋予了波粒二象性,\,粒子也要被理解为场,\,但所有场都要被被分解为平面波,\,平面波又要被量子化.\,如果没有相互作用,\,那么这些平面波的运动状态就不会发生改变,\,只有相互作用才会改变其运动,\,或传播方向发生偏折(发生散射),\,或强度发生改变(粒子的产生与湮灭).\,牛顿的观点是以\emph{相互作用}(interaction)为核心的观点.
	\item[拉格朗日,\,哈密顿等:]\, 将牛顿的结果与观点上升到\emph{分析力学}(analytic mechanics)的高度.\,将复杂体系的结构浓缩在统一的一个能量函数或作用量泛函数中.\,为后来场论建立和相对论,\,统计力学,\,量子力学的发展都提供了理论支持.
	\item[马赫:]\, 批判牛顿的过于绝对的时空观.\,预言质量分布会对惯性本身产生影响.
	\item[爱因斯坦开创的几何观:] \,  仿佛又回到了亚里士多德,\,不仅仅是无相互作用下的匀速直线运动,\,一切在万有引力下的天体运动其实也是``自然''的运动.\,看上去复杂的曲线运动不过也是沿对被引力所扭曲的弯曲时空几何下的天然``直线''---测地线(geodesic)运动而已.\,对引力问题提出开创性的解释.\,率先描绘了20世纪建立的若干理论的物理图像.
	\item[天体物理,\,粒子物理:]\, 爱因斯坦以后的理论研究中很多问题还处于开放状态.\,时而采用牛顿的相互作用观,\,时而采用爱因斯坦的几何观.\,是一种两个图像的有机结合.
\end{description}

我们今天说\emph{经典力学}(classical mechanics)大多数情况即指\emph{牛顿力学}(Newtonian mechanics),\,指的是由伽利略最初引领,\,笛卡尔,\,惠更斯等奠基,\,牛顿集大成的一套理论.\,它以伽利略的\emph{两种新科学的对话与数学展示}(Discorsi e dimostrazioni matematiche intorno a due nuove scienze)和牛顿的\emph{自然哲学的数学原理}(Philosophi\ae\ Naturalis Principia Mathematica)两篇文献为核心.\,逻辑上以质点的牛顿三大定律为基础.\,不包括后来上升到的分析力学层次,\,作为绝对时空观适用伽利略变换,\,不兼容相对论.

牛顿定律总结如下:
\vspace{0.5cm}

\hrule
\begin{enumerate}
	\item 第一定律:\quad 在惯性系中,\,一个质点应静止或匀速直线运动.\,除非有外力施加于物体.
	\item 第二定律:\quad 在惯性系中,\,质点受力的矢量和$\bs{F}$等于其质量$m$与加速度$\bs{a}$的乘积.
	\item 第三定律:\quad 一个质点对另一个质点施加作用力时,\,另一个质点同时将施加反作用力于第一个质点,\,作用力与反作用力等大反向共线.
\end{enumerate}
\hrule
\vspace{0.5cm}

三个定律之间的联系是很值得讨论的.\,对三个基本定律的设置反映了很多物理思想\footnote{例如爱因斯坦狭义相对论的两个假设在某种意义上与牛顿第一第二定律的逻辑完全一致.\,都是先定义时空背景再阐述物理规律的共性.}.\,分析如下:

\subsection{牛顿第一定律}




\section{动量定律}

\section{能量定律}

\section{角动量定律}

\section{位力定律*}

\section{动力学问题}

\section{碰撞问题}

