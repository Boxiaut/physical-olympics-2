%!TEX root = ../physical-olympics-2.tex
\chapter{动力学}


\section{牛顿定律}

\subsection{概述}
中世纪到近代以来物理学的发展,\,可谓沧海桑田,\,旧的观点被否定,\,几十年以后突然又以全新的面貌出现在物理学理论中.\,其中,\,实验中观察到各种令人惊诧的结果,\,无论物理学所依托的数学形式一变再变,\,要说亘古不变的主题,\,恐怕只有自然界的神奇与深邃,\,和我等凡人难以参透其中奥妙却仍为之孜孜不倦研究的决心.\,物理是一门研究自然界存在的运动并提出解释与理解方法的学科.\,运动形式多种多样,\,变化万千,\,但很多都被我们或多或少地解释,\,其实总结来看,\,从古至今我们也仅仅拥有过两类解释:\,一类由牛顿提出,\,一类则来自爱因斯坦.\,前前后后经历了大概以下发展:

\begin{description}
	\item[亚里士多德:]\, 天体运行,\,自由落体是自然(nature)的运动;\, ``树欲静而风不止''是强迫(violent)的运动.\,但错误地认为``物体越重,\,下落越快''
	\item[伽利略:]\, 辩证地看待亚里士多德的观点,\,开创性地提出相对性原理,\,成功论证与形成惯性与常见运动的正确认识.
	\item[笛卡尔,\,惠更斯等:]\, 成功得到大量动力学公式,\,包括后来的牛顿第二定律,\,动能的表达式与能量守恒定律,
	\item[牛顿开创的相互作用观:] \, 开创性的提出物质间存在相互作用,\,相互作用是改变物质运动状态的原因.\,并很好地把天体的运动归结到互相之间产生的相互作用上.\,后人建立了场的理论以后,\,超距的粒子间相互作用就不再被理论学家们采纳了,\,而要理解为场与粒子的相互作用,\,再晚一点,\,场和粒子都被赋予了波粒二象性,\,粒子也要被理解为场,\,但所有场都要被被分解为平面波,\,平面波又要被量子化.\,如果没有相互作用,\,那么这些平面波的运动状态就不会发生改变,\,只有相互作用才会改变其运动,\,或传播方向发生偏折(发生散射),\,或强度发生改变(粒子的产生与湮灭).\,牛顿的观点是以\emph{相互作用}(interaction)为核心的观点.
	\item[拉格朗日,\,哈密顿等:]\, 将牛顿的结果与观点上升到\emph{分析力学}(analytic mechanics)的高度.\,将复杂体系的结构浓缩在统一的一个能量函数或作用量泛函数中.\,为后来场论建立和相对论,\,统计力学,\,量子力学的发展都提供了理论支持.
	\item[马赫:]\, 批判牛顿的过于绝对的时空观.\,预言质量分布会对惯性本身产生影响.
	\item[爱因斯坦开创的几何观:] \,  仿佛又回到了亚里士多德,\,不仅仅是无相互作用下的匀速直线运动,\,一切在万有引力下的天体运动其实也是``自然''的运动.\,看上去复杂的曲线运动不过也是沿对被引力所扭曲的弯曲时空几何下的天然``直线''---测地线(geodesic)运动而已.\,对引力问题提出开创性的解释.\,率先描绘了20世纪建立的若干理论的物理图像.
	\item[天体物理,\,粒子物理:]\, 爱因斯坦以后的理论研究中很多问题还处于开放状态.\,时而采用牛顿的相互作用观,\,时而采用爱因斯坦的几何观.\,是一种两个图像的有机结合.
\end{description}

我们今天说\emph{经典力学}(classical mechanics)大多数情况即指\emph{牛顿力学}(Newtonian mechanics),\,指的是由伽利略最初引领,\,笛卡尔,\,惠更斯等奠基,\,牛顿集大成的一套理论.\,它以伽利略的\emph{两种新科学的对话与数学展示}(Discorsi e dimostrazioni matematiche intorno a due nuove scienze)和牛顿的\emph{自然哲学的数学原理}(Philosophi\ae\ Naturalis Principia Mathematica)两篇文献为核心.\,逻辑上以质点的牛顿三大定律为基础.\,不包括后来上升到的分析力学层次,\,作为绝对时空观适用伽利略变换,\,不兼容相对论.

牛顿定律总结如下:
\vspace{0.5cm}

\hrule
\begin{enumerate}
	\item 第一定律:\quad 在惯性系中,\,一个质点应静止或匀速直线运动.\,除非有外力施加于物体.
	\item 第二定律:\quad 在惯性系中,\,质点受力的矢量和$\bs{F}$等于其质量$m$与加速度$\bs{a}$的乘积.
	\item 第三定律:\quad 一个质点对另一个质点施加作用力时,\,另一个质点同时将施加反作用力于第一个质点,\,作用力与反作用力等大反向共线.
\end{enumerate}
\hrule
\vspace{0.5cm}

三个定律之间的联系是很值得讨论的.\,对三个基本定律的设置反映了很多物理思想\footnote{例如爱因斯坦狭义相对论的两个假设在某种意义上与牛顿第一第二定律的逻辑完全一致.\,都是先定义时空背景再阐述物理规律的共性.}.\,分析如下:

\subsection{牛顿第一定律}
牛顿第一定律旨在阐述时空的平直性.

也正基于以上这一点,\,我们不能仍为牛顿第一定律可以由牛顿第二定律出发推导.\,它不是表面上的牛顿第二定律中$\bs{F}=\bs{0},\,\bs{a}=\bs{0}$的特例.\,恰恰相反,\,即使从形式上看,\,也应当先由牛顿第一定律给出\emph{惯性系}(inertial frame)的概念才能适用牛顿第二定律.\,那么怎样的参考系是惯性系呢?\,牛顿第一定律是暗含认可\emph{绝对惯性系}(absolute inertial frame)的存在性的.\,在这样第一个系中牛顿第一定律才可以成立.\,这一个系无非就是由某种特殊运动状态的观察者(绝对观察者)建立的三维欧几里得空间坐标系外带一维时间坐标.\,而只要质点不受力(力是对相互作用的描述,\,只要远离其它所有物质就会使质点成为\emph{孤立}(isolated)的),\,就会静止或匀速直线运动,\,所谓匀速直线运动是指:
\[\begin{pmatrix}x(t)\\y(t)\\z(t)\end{pmatrix}=\begin{pmatrix}x(0)\\y(0)\\z(0)\end{pmatrix}+\begin{pmatrix}v_x\\v_y\\v_z\end{pmatrix}\cdot t\]

其中三个速度$v_x,\,v_y,\,v_z$都是常数.\,但是我们发现,\,相对这个系做匀速直线运动的另一个无自转观察者建立的系中,\,原来做匀速直线运动的质点现在依然做匀速直线运动.\,所以这样的系也叫做惯性系,\,可以叫做相对惯性系.\,即:\,惯性系就是任何相对绝对惯性系做匀速直线平动的参考系.\,而惯性系就是牛顿定律成立的前提条件.

所以牛顿的经典力学体系的第一大特点是承认绝对惯性系的存在性.\,实际上承认这一点也几乎等价于承认了时空的平直性.\,只不过在众多彼此等价(任何物理实验都不足以判断这些系的不等价性)的系中,\,牛顿认为还应当有一个第一性的,\,``最初''的惯性系.

\subsection{牛顿第二定律}
牛顿第二定律是牛顿定律的核心,\,旨在描述相互作用对运动的影响.

\emph{相互作用}(interaction)是牛顿经典力学后经过了几百年发展而浓缩出来的更普遍的概念.\,如果没有相互作用,\,物质基本的运动模式就是``惯性''的,\,如匀速运动的粒子或传播的平面波.\,任何使物质偏离这种``惯性''运动的模式的原因就都在相互作用上.\,相互作用的模式是多种多样的,\,除了牛顿指出的下面仔细讨论的情形,\,光会因为与物质相互作用而产生吸收和散射,\,两个惰性气体原子靠近时两个电子云也会产生量子的诱导极化而使原子间产生范德瓦耳斯吸引.\,这些都是广义的相互作用,\,但是为了描述它们,\,前者需要借助波来描述光这种物质,\,后者还需要借助波函数来描述电子云,\,相互作用更是要依赖于薛定谔方程等来理解.\,所以经典物理理论有它的适用性,\,这一点在下面我们需要记住.

这也就是牛顿经典力学体系的第二大特点了:\,物质的基本模型一律视作\emph{质点}(mass point)\footnote{牛顿甚至认为光也适用于质点模型},\,相互作用的基本模型一律视作\emph{力}(force),\,质点具有惯性这一基本属性,\,用\emph{质量}(mass)来描述.

从牛顿第二定律$F=ma$形式上看,\,难免会由于数学思维得出``用力来定义质量的多少''或者是``用质量定义力的大小''这样的观点,\,而进入力与质量循环定义的误区.\,但实际上谁也没有定义谁.\,因为正常的物理思维是先要提前定义质量和力:

首先质量是受力物体所固有的属性之一,\,它具有广延性:\,如果把质量为$m_1$的质点与质量为$m_2$的质点复合为一个大质点,\,其质量变为$m_1+m_2$.\,可比性:\,任意两个质点的质量都可以依照在同样大小的力下的行为来比较其质量的比例关系\footnote{事实上用碰撞实验可以给出一个更简单的比较质量的方法,\,让待测质量的质点以一个单位的速度去正碰撞一个以$x$个单位速度运动的单位质量的质点,\,若恰好停下就说明待测质量为$x$个单位质量,\,若没停下就改变$x$的值直到停下.}.\,和固有性:\,只要质点没有变性,\,质量就不变,\,所以在不同参考系去看同一个质点,\,在运动的不同时刻质量都是严格不变的,\,也称``质量守恒'',\,我们将避免这一称呼,\,因为一般的守恒量只是针对一个参考系中的过程中的各个状态,\,而换一个系看守恒量往往要变,\,质量却显然不变,\,这是其一;\,另外相对论下质量就不再守恒了,\,它不普遍,\,这是其二.

力也是对固有的相互作用的强度做的描述.\,它也是固有的,\,应当有一种合理的公式去描述施力物体的状态如何影响施加的力(这个力也与受力物体的状态有关).\,而且如果换一个参考系考虑,\,力的强度是不变的.\,根据力对质点产生的效果,\,力也应当具有矢量性,\,不同两个力是可以比较的,\,很多情况下甚至可以叠加:\,多个施力物体互不影响各自施加的力,\,同时作用于同一个受力物体,\,在生活中这种现象非常常见,\,不难得出这样的总结.

于是才有了牛顿的著名论断:
\[\bs{F}=m\bs{a}\]

这样一个定律内容是丰富的,\,它首先是兼容于伽利略相对性原理从而自洽:\,在另外的惯性系中考察,\,$F$和$a$都是不变的.\,但是又不是兼容于相对性原理的唯一解决方案,\,狭义相对论,\,就提供了另一套可能性,\,在低速极限$v\ll c$时狭义相对论的动力学方程表达式和这并无二致;\,其次质量$m$作为系数联系了力(原因)与加速度(结果)这并不是偶然,\,它也是为了和叠加原理相协调:\,一份的力作用在一份质点上产生这样的加速度,\,那么把两份力各自作用在两份质点上那就还是产生同样的加速度.\,所以力除以质量决定运动学加速度\footnote{即$F/m=f(v,\,a)$}那是天经地义.\,至于为什么两份力作用在原来的一份质点上会产生加速度加倍的效果,\,这就是经典力学中的类似于公理性质的存在了.

\subsection{牛顿第三定律}
牛顿第三定律是对``相互作用''中的``相互性''的刻画,\,暗示了动量与角动量守恒.

任何物理理论,\,如果妄想在短短几条式子中道尽万千世界发展与变化的规律,\,都是玄学而不是科学.\,科学的精神是要基于事实,\,进行总结与归纳,\,借助数学进行基本的抽象与再创造.\,所以现实就是,\,世界的运行本就是复杂的,\,不可预测的.\,物理学的发展史也就充满了惊人的转折:\,每当人们以为在某一个大的领域有简单而清晰的图像共识的时候,\,总是会有意外发生告诉人们特殊情况下理论不再适用.\,原则上,\,没有任何命题能够成为物理学上的公理:\,它能无条件适用于从宇观到微观的方方面面.

牛顿清楚地意识到了物质与物质之间相互作用的复杂性,\,是不可能用几个简单的公式写出所有情况下他的理论中两个任意质点之间的作用力的.\,所以退而求其次,\,用牛顿第三定律来描述了所有情形下这样的力需要满足的条件.\,的确合情合理.\,但是又看牛顿第三定律的提法本身是否合理?

实际上我们先注意到动量守恒和角动量守恒的条件比牛顿第三定律的条件要强(也更本质):\,我们可以由前者推导出后者.\,以动量为例,\,动量是用来描述一个体系的动力学运动趋势的.\,动量守恒就意味着,\,如果一个孤立的体系的一部分``平均地''在初始时刻具有向前运动的趋势,\,比如一部分向前运动而另一部分静止.\,那么之后就绝对不可能演化为向后运动.\,或者还可以这么理解:\,如果看到一个质点在做一个非匀速直线运动:\,比如忽快忽慢的直线运动,\,那么显然单独看质点本身动量并不守恒.\,但是这必然意味着质点不是孤立的,\,必须和别的体系一同看作一个整体,\,质点增加或减少的动量必须由体系的另一部分的动量减少或增加来抵消,\,来满足整体动量守恒的需求.\,如果体系仅仅由两个质点构成,\,再把动量的增加或减少的快慢理解为受力(动量定理),\,我们就有了牛顿第三定律中关于相互作用力等大,\,反向的部分.

所以动量守恒可以用来``证明''牛顿第三定律中的一对相互作用力.\,同理角动量守恒则可以用来证明牛顿第三定律中质点间相互作用力共线的部分.\,最后读者可以验证牛顿第三定律的数学形式为\footnote{$\bs{F}_{12}$表示1给2的力.}:
\[\bs{F}_{12}+\bs{F}_{21}=\bs{0}\quad,\quad \bs{r}_2\times\bs{F}_{12}+\bs{r}_1\times\bs{F}_{21}=\bs{0}\]

之所以这么理解是因为,\,动量守恒和角动量守恒显得更普适.\,事实上,\,它们是体系物理规律在空间平移对称性和旋转对称性下具有协变形式的直接推论.\,这一点是由近代著名的\emph{诺尔特定理}(Noether Theorem)这一数学定理精确证明了的.

也正是牛顿第三定律暗示了牛顿经典力学体系的缺点.\,因为我们无法解释牛顿第三定律的``自然性'',\,取消牛顿第三定律所构成的体系无非是具有外力的非孤立系(动量不守恒),\,牛顿第三定律为体系增添的``结构性要求''并没有很好地由理论解释与包含.\,所以才有了后来建立在对守恒量,\,尤其是能量进行强调的分析力学理论的提出.

\subsection{质点系与它的牛顿定律}
对于复杂体系,\,牛顿提出一种可能的等效观点,\,就是把任意这样的体系看做是一种离散的质点系的推广.\,质点系由质点集合$\{(m_i,\,\bs{r}_i)\},\,(i=1,\,2\,\cdots \,n)$组成.\,我们只要找到从质点理论到有限质点构成的质点系理论中那些不变的结论,\,就有理由相信这些结论可以推广到复杂的真实体系.\,因为真实体系被期待可以用一个足够大数目的质点系来近似.\,对于复杂的质点系首先需要引入\emph{质心}(center of mass,\,{\rm CM})的概念:
\[\overline{\{(m_i,\,\bs{r}_i)\}}=(m_C,\,\bs{r}_C)\]

我们的做法是:\,直接把质心理解为一个新的质点,\,它具有新的质量$m_C$和位置$\bs{r}_C$,\,并满足表达式:
\[m_C=\sum_i m_i\quad;\quad \bs{r}_C=\frac{\sum_i m_i\bs{r}_i}{\sum_i m_i}\]

可以把上式读作:\,质心的质量等于质点系的总质量,\,质心的位矢是各个质点位矢关于质量的加权平均.

质心的求法是符合交换律与结合律的.\,这在质量上是显然的,\,因为质心质量就是各个质点质量构成的连加式.\,对于质心位矢,\,考虑总质量为$m_1$的第一个质点系$(m_{1i},\,\bs{r}_{1i})$和总质量为$m_2$的第二个质点系$(m_{2j},\,\bs{r}_{2j})$复合而成的整个质点系,\,如果直接算两个体系质心形成的质心:
\[\frac{m_1\bs{r}_{1C}+m_2\bs{r}_{2C}}{m_1+m_2}=\frac{\displaystyle\sum_i m_{1i}\cdot \frac{\displaystyle\sum_{i} m_{1i}\bs{r}_{1i}}{\displaystyle\sum_i m_{1i}}+\displaystyle\sum_j m_{2j}\cdot\frac{\displaystyle\sum_{j} m_{2j}\bs{r}_{2j}}{\displaystyle\sum_{j} m_{2j}}}{\displaystyle\sum_i m_{1i}+\displaystyle\sum_j m_{2j}}=\frac{\displaystyle\sum_{k\in 1\& 2}m_k\bs{r}_k}{\displaystyle\sum_{k\in 1\& 2}m_k}\]

可以发现这就是整个体系的质心.\,这种方法就是``组合法'',\,比如求几根棍子的质心,\,可以把棍子替换为在质心的质点再求.\,由此还可以引申出``割补法''或者``负质量法'',\,不过就是把上式反过来理解:\,如果我们要求第一个体系的质心,\,不难验证可以这样求:
\[\bs{r}_{1C}=\frac{m\bs{r}_C-m_2\bs{r}_{2C}}{m-m_2}\]

动态地看待求质心的式子,\,还可以发现,\,求导以后,\,我们得到了质心作为一个在运动的质点,\,其速度,\,加速度也是可以直接从原质点系来做加权平均的:
\[\bs{v}_C=\frac{\sum_i m_i\bs{v}_i}{\sum_i m_i}\quad;\quad \bs{a}_C=\frac{\sum_i m_i\bs{a}_i}{\sum_i m_i}\]

为了推导出质点系的类似的牛顿定律,\,我们发现牛顿三大定律是缺一不可的.\,当然原则上如果没有牛顿第三定律,\,我们就把作用在每一个质点$m_i$上的合力,\,不管施力物体是谁,\,记作$\phantom{}^tF_i$,\,``t'' for ``total''.\,原则上我们也可以得到:
\[\phantom{}^t\bs{F}=\sum_i m_i\bs{a}_i\quad \phantom{}^t\bs{F}=\sum_i\phantom{}^t\bs{F}_i\]

这一点很容易证明,\,只需要把每个质点的牛顿定律加起来即可.\,但是注意到牛顿第三定律既然正确,\,我们就可以做进一步的研究:\,把任何质点受力分解为\emph{内力}(internal force)和\emph{外力}(external force):
\[\phantom{}^t\bs{F}_i=\phantom{}^{in}\bs{F}_i+\phantom{}^{ex}\bs{F}_i\]

内力指$i=1,\,2\,\cdots\,n$个质点之间一对一对存在的力,\,所以实际上不如写明:
\[\phantom{}^{in}\bs{F}_i=\sum_{j\neq i}\bs{F}_{ji}\]

最后单独的那个真正的外力以后直接简写为$\phantom{}^{ex}\bs{F}_i=\bs{F}_i$.

经过这么一番折腾完以后,\,我们还是把所有质点的牛顿定律来求和,\,但是注意到由于牛顿第三定律关于相互作用力等大反向的部分,\,内力相互抵消,\,得到:
\[\bs{F}=\sum_i m_i\bs{a}_i\quad \bs{F}=\sum_i \bs{F}_i\]

这就是质点系的牛顿第二定律,\,相比上面那个也对的直接把每个牛二加起来的表达式,\,能引起注意的变化只有一点,\,那就是:\,不需要考虑内力!\,但正是这一点极大地简化了很多实际问题的分析.

值得一提,\,我们顺理成章地要考虑另外两个层面的规律,\,一是``对质心'',\,二是``相对质心''.\,前者值的是,\,把质心作为一个质点来研究问题.\,那么问题也很简单,\,利用之前证明过的:
\[\bs{a}_C=\frac{\sum_i m_i\bs{a}_i}{\sum_i m_i}\quad \Rightarrow \quad m_C\bs{a}_C=\sum_i m_i\bs{a}_i\]

就可以得到:
\[\bs{F}=m_C\bs{a}_C\]

这就是著名的\emph{质心运动定律}(motion law of {\rm CM}).\,特殊的,\,如果合外力$\bs{F}=\bs{0}$.\,那么退化到\emph{质心守恒}:\,质心做匀速直线运动或者静止.

最后我们提出质心系的概念.\,\emph{质心系}(frame relative to {\rm CM})指:\,原点在质心的平动欧几里得坐标系.\,结合我们已经学过的和将来要学的所有内容我们把其性质罗列如下,\,请读者能够给出证明的直接证明,\,给不出的带着问题往下学习:
\begin{itemize}
	\item 质心系虽然有非惯性力,\,但非惯性力与原力系主矢和为零.
	\item 质心系虽然有非惯性力,\,但非惯性力系对质心的主矩为零.
	\item 质心系是零动量系.
	\item 质心系虽然有非惯性力,\,但非惯性力在过程中总功和为零.
	\item 取质心为原点能使质点系二阶矩最小,\,故刚体相对所有平行轴中以过质心的轴转动惯量最小.
	\item 只有取质心才能写出柯尼希定理.
\end{itemize}

\subsection{非惯性系的处理}
虽然牛顿定律只在惯性系中成立.\,但是如果我们选取一个相对惯性系的转动系,\,其原点速度加速度为$\bs{v}_C$与$\bs{a}_C$.\,而角速度与角加速度分别为$\bs{\omega}$与$\bs{\beta}$.\,本来在原来的惯性系中研究一个质点,\,正确的式子是:
\[\bs{F}=m\bs{a}\]

现在换了参考系以后,\,将不能仿照上式写出一个依然正确的式子.\,但是加速度变换总是对的:
\[\bs{a}=\bs{a}_C+\bs{\beta}\times\bs{r}'+\bs{\omega}\times(\bs{\omega}\times\bs{r}')+2\bs{\omega}\times\bs{v}'+\bs{a}'\]

而我们也永远认为力矢量在参考系变化下是大小与方向都不变的:
\[\bs{F}=\bs{F}'\]

代入就足以得到依然正确的,\,仅仅用新的非惯性系中的各个参量表示的类牛顿定律.\,我们以后将直接叫做非惯性系中的牛顿定律:
\[\bs{F}'+(-m\bs{a}_C)+(-m\bs{\beta}\times\bs{r}')+[-m\bs{\omega}\times(\bs{\omega}\times\bs{r}')]+(-2m\bs{\omega}\times\bs{v}')=m\bs{a}'\]

可以发现,\,这类似于需要为质点运动虚构一些力,\,它们全都称作\emph{惯性力}(force of inertial).\,包括:
\begin{itemize}
	\item 平动惯性力$-m\bs{a}_C$,\,典型的情况就是平动而非转动的非惯性系中的惯性力.\,特点是一个恒力,\,而与质点的运动状态无关.
	\item 切向惯性力$-m\bs{\beta}\times\bs{r}'$与惯性离心力$-m\bs{\omega}\times(\bs{\omega}\times\bs{r}')$,\,典型的情况是绕轴做旋转的参考系.\,特点是这两个力仅仅依赖于质点的位置,\,离轴越远力越大,\,在轴上力就消失了.
	\item 柯里奥利力$-2m\bs{\omega}\times\bs{v}'$,\,典型的情况是相对转动系还有运动的情况.\,特点是给出一个与位置无关,\,但是正比与速度大小,\,方向始终垂直于速度方向的力.
\end{itemize}

如果我们的研究对象是一个刚体而非质点,\,那么相当于每一个质点上都要受到不同的惯性力.\,此时如果考虑这些所有力作用在刚体上的等效合力和力矩,\,那么将是一个非常复杂的情况,\,尤其是,\,各个柯里奥利力将对刚体整体造成所谓的``回转力矩'',\,用这个观点可以解释很多第七章刚体中的现象,\,感兴趣的读者可以参考相关资料.

\section{动量定律}

\subsection{质点的动量}

\subsection{质点系的动量}


\section{角动量定律}

\subsection{质点的角动量}

\subsection{质点系的角动量}


\section{能量定律}

\subsection{质点的能量}

\subsection{质点系的能量}


\section{位力定律*}

\subsection{质点的位力}

\subsection{质点系的位力}



\section{动力学问题}

\section{碰撞问题}

